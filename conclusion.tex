% !TeX encoding=utf8
% !TeX spellcheck = en-US

\section{Conclusions and Future Work}
\label{sec:con}

In this work, we discussed the implementation of an EaaS framework within OrganiCity, and have presented our position regarding EaaS, i.e., to use existing IoT deployments in multiple cities in a federated manner to implement smart city prototype solutions. In this context, we presented our toolset, which enables EaaS in this application domain, in a scaled and integrated manner that has not been implemented previously. We have presented our design and implementation, discussed how OrganiCity and its toolset fit into the current smart city landscape, and presented several core components.

In order to validate this approach, we discussed learnings from deploying an EaaS framework in the wild, by running 25 experiments from independent experimenter teams. We also presented two specific experiments utilizing the OC platform as well as elaborated and discussed specific experimenter learnings, produced by using the first version of our platform. Our first findings indicate that this toolset has been utilized by the community and can be impactful. We discussed the ways in which the platform was used, where it fell short and how it has evolved through experimenter feedback. We additionally discussed results concerning the utilized architecture and scalability concerns, having used a more centralized architecture. From an architectural perspective, we can conclude that federation alternatives exist to handle potential scalability issues. In addition, following these approaches, any modification would be transparent to the systems implemented within the OC platform. We believe that this feedback, produced by such a scale of experimentation, will be very useful to the community currently working on similar aspects in the smart city domain.

OC is currently nearing its completion. The OC platform has been significantly re-factored taking experimenter learnings into account, and the updated OC platform has been put to the test in a second open call following the same approach as in the first one. Results concerning this round of experiments are not included in this article, but will be presented in our future work. 
