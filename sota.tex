% !TeX encoding=utf8
% !TeX spellcheck = en-US

\section{Related Work}
\label{sec:sota}

In terms of experimentation testbeds, SmartSantander~\cite{sanchez2014smartsantander} built one of the largest city-scale IoT research infrastructures, pioneering the experimentation of novel smart city architectures, services, and applications in real-world urban environments. It emphasized managing experiments at an IoT device level, while allowing data-acquiring tasks facilitating urban services on top of the captured data flows. SmartSantander built interfaces with FIWARE and FI-Lab\footnote{https://lab.fiware.org}, to support interconnectivity with the IoT/Future Internet community. WISEBED~\cite{Coulson2012} pioneered discrete IoT device testbed federation. 

Additionally, projects like IoT-Lab\footnote{http://www.iotlab.eu} investigate crowdsourcing and IoT services for supporting multidisciplinary research tasks. Their approach differs from ours, since their services are not tightly coupled with a smart city testbed. Festival~\cite{Akiyama2017} is another example of existing federating experimentation testbeds from Europe and Japan, in order to provide a unified infrastructure to the research community. However, its focus is not exclusively on the smart city domain, and it does not offer the scale or the toolset provided by OC as a platform overall. CPaaS.io~\cite{FogFlow} is an ongoing project with similar goals to OC. Synchronicity\footnote{http://synchronicity-iot.eu} features open calls for developing new services, similar to OC, but places a much larger focus on open data markets in the context of a smart city, leaving aside the co-creative approach.

In light of these advancements, OC aims to combine the aforementioned approaches and co-create new smart city solutions with citizens, researchers and city authorities. Crowdsensing (i.e., tasking groups of volunteers to gather various kinds of data using IoT devices and/or smartphones) is one of the directions taken to address this challenge, helping to build a smart city data repository. Regarding the OC platform, a more general discussion is provided in~\cite{gutierrez1}, describing its architecture and overall software stack, with potential co-creation capabilities showcased in~\cite{gutierrez2} in detail. Apart from aiming to provide a pragmatic solution to the crowdsensing problem, the toolset discussed in this work allows end-users to benefit from this interoperability in various ways (data storage, visualization, interfacing to other systems, community management, knowledge extraction and urban service creation), and not just basic management of crowdsensing activities.

As an example of this approach, OC's crowdsensing tool allows for a broad set of opportunities for integrating smartphones in a smart city experimentation context, allowing the use of such devices to produce experimental data as an extension of an existing IoT infrastructure. %We allow for easy interconnection to other IoT devices, adding further possibilities to the system. 
The mobile crowdsensing paradigm and the associated features and challenges are further discussed in~\cite{%crowdsensing-guo, 
crowdsensing-guo2}. With respect to incentives and crowdsensing task assignment, which are also part of our work, in~\cite{aliens, incentives-survey} such aspects are discussed in detail. In addition, \cite{balestrini-orchestration, ledantec} provides a discussion related to several research questions we are trying to answer in OC as well. Finally, as discussed in \Cref{sec:intro}, OC has managed to appeal to a large number of research teams so far, and has already produced a large number of urban experiments, far surpassing the respective numbers kick-started by other similar projects.
